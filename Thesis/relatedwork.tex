\chapter{Related Work}\label{relatedwork}
State-space dimensionality reduction has been a topic of interest in RL for several years. Curran et al. \cite{mario} used PCA to reduce the dimensionality of the state-space in a Super Mario environment. They found that with the right number of principal components, an agent using PCA was able to converge to a better policy and need less episodes than an agent using the full observation. Unlike the Starcraft II environment used in our research having image/grid based observations, their states and observations are comprised of non-spatial related variables; to, for instance, denote the presence of one or more enemies within one gridcell distance of Mario's position, they use one variable with $2^8$ possible values ($0-255$). Hence we are using PCA in a different setting, with an observation format more commonly used in modern RL. Furthermore they use the Q-learning algorithm (mentioned in section \ref{pl-dqn}) which is not a state-of-the-art learning algorithm anymore.

Research has also been done with regards to using autoencoders for state-space-dimensionality reduction in RL. Lange and Riedmiller in 2010 \cite{AE_2010} used visual data as observations and found that the agent was able to find an optimal policy using lower dimensional data from the autoencoder. They did not compare the performance of this agent with any other agent. Furthermore, their system only has $31$ possible states, which is very limited; in contrast, our environment has a total of $731.187$ possible states. 

In 2016, Van Hoof et al. \cite{AE_2016} used an autoencoder to project noisy sensor data unto a lower dimensional space. They found that the agent using the autoencoder was far better able to find a good policy than the agent using full observations. This was explained as the agent being too sensitive to noisy data, hence being unable to train even decently. Our research in contrast explores an environment where the baseline agent is able to train to a good policy.

In 2019, Prakash et al. \cite{AE_2019} also used an autoencoder on visual data to project the observation data onto a lower dimensional space. They found that an agent using the autoencoder far outperformed the baseline agent. Again though, the baseline agent did not find a decent policy, though the authors claim that with enough episodes it would have.

Another paper in 2019 by Gelada et al. \cite{deepmdp} compared using a DeepMDP with using an autoencoder for state-space dimensionality reduction. They found that in simpler environments a DeepMDP can find better representations on lower dimensionality space than an autoencoder. Simultaneously though, they also find that in more complex environments, like Atari games, DeepMDPs can have trouble finding a good representation and that its loss can be difficult to optimize. They also find that a DeepMDP agent generally outperforms a baseline agent. 

%TODO demonstrating that you have found a \emph{new} solution / approach / method?
