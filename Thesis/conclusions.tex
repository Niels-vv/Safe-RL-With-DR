\chapter{Conclusions and future work}\label{conclusions}
In this chapter you present all conclusions that can be drawn from the
preceding chapters.
It should not introduce new experiments, theories, investigations, etc.:
these should have been written down earlier in the thesis.
Therefore, conclusions can be brief and to the point.

\emph{future work}
- ae met lagere dimensions vergelijken
- env zonder rgb maar directe features pakken voor betere pca vergelijking
- voordelen dim red bekijken: nu focus op haalbaarheid van dim red, dus zoveel mogelijk gelijek agent architectures. Maar kan kijken naar verglijkingen tussen compleet vershcillende agents als gevolg van dim red (dus bv kijken training tijd en training eps omlaag).
- ae met grotere CNN architecture die pas in latere layers downsampled zodat je minder spatial information verliest